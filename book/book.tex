\documentclass{book}
\usepackage[utf8]{inputenc}
\usepackage[english]{babel}
\usepackage{longtable}
\usepackage{enumerate} 
\usepackage{booktabs}
\usepackage{hyperref}
\usepackage[dvipsnames]{xcolor}
\usepackage{listings}       % via http://tex.stackexchange.com/questions/121601/automatically-wrap-the-text-in-verbatim 
\lstset{
basicstyle=\small\ttfamily,
columns=flexible,
breaklines=true
}
\usepackage{minted}
\usemintedstyle{borland}

\renewcommand{\familydefault}{\sfdefault}

\makeatletter
\newcommand\href@footnote[2]{#2\footnote{\url{#1}}}
\DeclareRobustCommand{\href}{\hyper@normalise\href@footnote}
\makeatother

\def\tightlist{}

\hypersetup{
    bookmarks=true,         % show bookmarks bar?
    unicode=false,          % non-Latin characters in Acrobat’s bookmarks
    pdftoolbar=true,        % show Acrobat’s toolbar?
    pdfmenubar=true,        % show Acrobat’s menu?
    pdffitwindow=false,     % window fit to page when opened
    pdfstartview={FitH},    % fits the width of the page to the window
    pdftitle={Daily Programming Challenges from Reddit},    % title
    pdfauthor={Jose Nazario},     % author
    pdfsubject={Computer Programming},   % subject of the document
    pdfcreator={LaTeX},   % creator of the document
    pdfproducer={Producer}, % producer of the document
    pdfkeywords={keyword1, key2, key3}, % list of keywords
    pdfnewwindow=true,      % links in new PDF window
    colorlinks=false,       % false: boxed links; true: colored links
    linkcolor=red,          % color of internal links (change box color with linkbordercolor)
    citecolor=green,        % color of links to bibliography
    filecolor=magenta,      % color of file links
    urlcolor=cyan           % color of external links
}

\title{Daily Programming Challenges from Reddit}

\begin{document}
    
\maketitle

\tableofcontents

\chapter{Easy}

\section*{Introduction}

Easy challenges are meant to be fun and a great way to get into programming or a new language. For myself, I think of easy challenges as ones we can see the general idea of how to solve, but we may not know our way around a language yet to tackle it. The easy challenge gives us a concrete way to tackle it. 

As you get more proficient, you'll find these can be tackled in just a few minutes in your language of choice. However they also make a great tool for learning a new language no matter how experienced you are. Many, but not all, can be solved in one or two functions. 

Often in the /r/dailyprogrammer community we'll see more experienced programmers give advice to younger ones about doing things more clearly or in a more idiomatic style. This is great feedback.

\input easy.tex

\chapter{Intermediate}

\section*{Introduction}

Intermediate challenges are meant to help you stretch your chops in your language of choice. You may have to dig around a few esoteric corners of your language, but are still just a programming exercise - they have a clear path to a solution in whatever language you choose. 

\input int.tex

\chapter{Hard}

Hard challenges are meant to really get you thinking like a computer scientist. Often I choose NP hard problems because they force you to consider time-space tradeoffs in a strive for efficiency. These not only require a confidence in a language but often the skills needed to be a good programmer - laying out the problem clearly and seeing a path to a solution, often with a pencil and paper, before you begin coding. 

\input hard.tex

\end{document}
