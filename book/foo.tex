\documentclass{article}
\usepackage[utf8]{inputenc}
\usepackage[english]{babel}
\usepackage{hyperref}
\usepackage[dvipsnames]{xcolor}
\usepackage{minted}
\usemintedstyle{borland}


\hypersetup{
    bookmarks=true,         % show bookmarks bar?
    unicode=false,          % non-Latin characters in Acrobat’s bookmarks
    pdftoolbar=true,        % show Acrobat’s toolbar?
    pdfmenubar=true,        % show Acrobat’s menu?
    pdffitwindow=false,     % window fit to page when opened
    pdfstartview={FitH},    % fits the width of the page to the window
    pdftitle={My title},    % title
    pdfauthor={Author},     % author
    pdfsubject={Subject},   % subject of the document
    pdfcreator={Creator},   % creator of the document
    pdfproducer={Producer}, % producer of the document
    pdfkeywords={keyword1, key2, key3}, % list of keywords
    pdfnewwindow=true,      % links in new PDF window
    colorlinks=false,       % false: boxed links; true: colored links
    linkcolor=red,          % color of internal links (change box color with linkbordercolor)
    citecolor=green,        % color of links to bibliography
    filecolor=magenta,      % color of file links
    urlcolor=cyan           % color of external links
}

\begin{document}

\section{Title}\label{title}

Anagram Detector

\section{Difficulty}\label{difficulty}

Easy

\section{Description}\label{description}

An anagram is a word (or set of words) that use the same letters just
rearranged. All words must be valid spelling. Someone once said, ``All
the life's wisdom can be found in anagrams. Anagrams never lie.'' How
they don't lie is beyond me, but there you go. Punctuation and
capitalization don't matter.

\section{Input Description}\label{input-description}

You'll be given two words or sets of words separated by a question mark.
Your task is to replace the question mark with information about the
validity of the anagram. Example:

\begin{verbatim}
"Clint Eastwood" ? "Old West Action"
"parliament" ? "partial man"
\end{verbatim}

\section{Output Description}\label{output-description}

You should replace the question mark with some marker about the validity
of the anagram proposed. Example:

\begin{verbatim}
"Clint Eastwood" is an anagram of "Old West Action"
"parliament" is NOT an anagram of "partial man"
\end{verbatim}

\section{Challenge Input}\label{challenge-input}

\begin{verbatim}
"wisdom" ? "mid sow"
"Seth Rogan" ? "Gathers No"
"Reddit" ? "Eat Dirt"
"Schoolmaster" ? "The classroom"
"Astronomers" ? "Moon starer"
"Vacation Times" ? "I'm Not as Active"
"Dormitory" ? "Dirty Rooms"
\end{verbatim}

\section{Challenge Output}\label{challenge-output}

\begin{verbatim}
"wisdom" is an anagram of "mid sow"
"Seth Rogan" is an anagram of "Gathers No"
"Reddit" is NOT an anagram of "Eat Dirt"
"Schoolmaster" is an anagram of "The classroom"
"Astronomers" is NOT an anagram of "Moon starer"
"Vacation Times" is an anagram of "I'm Not as Active"
"Dormitory" is NOT an anagram of "Dirty Rooms"
\end{verbatim}

\section{Scala Solution}\label{solution-1}
\begin{minted}
    [
    framesep=2mm,
    baselinestretch=1.2,
    bgcolor=lightgray,
    fontsize=\footnotesize,
    linenos
    ]
    {javascript}
// returns true if the two words are an anagram, false if otherwise
def anagram(s1:String, s2:String): Boolean = 
    s1.toLowerCase().filter(_.isLetter).sorted == \
      s2.toLowerCase().filter(_.isLetter).sorted
\end{minted}

\section{Title}

Concatenated Integers

\section{Difficulty}\label{difficulty-1}

Easy

\section{Description}\label{description-1}

Given a list of integers separated by a single space on standard input,
print out the largest and smallest values that can be obtained by
concatenating the integers together on their own line. This is from
\href{https://blog.svpino.com/2015/05/07/five-programming-problems-every-software-engineer-should-be-able-to-solve-in-less-than-1-hour}{Five
programming problems every Software Engineer should be able to solve in
less than 1 hour}, problem 4. Leading 0s are not allowed (e.g.~01234 is
not a valid entry).

\section{Sample Inputl}\label{sample-inputl}

You'll be given a handful of integers per line. Example:

\begin{verbatim}
5 56 50
\end{verbatim}

\section{Sample Output}\label{sample-output}

You should emit the smallest and largest integer you can make, per line.
Example:

\begin{verbatim}
50556 56550
\end{verbatim}

\section{Challenge Input}\label{challenge-input-1}

\begin{verbatim}
79 82 34 83 69
420 34 19 71 341
17 32 91 7 46
\end{verbatim}

\section{Challenge Output}\label{challenge-output-1}

\begin{verbatim}
3469798283 8382796934
193413442071 714203434119
173246791 917463217
\end{verbatim}

\section{Scala Solution}\label{scala-solution}

\begin{minted}
    [
    framesep=2mm,
    baselinestretch=1.2,
    bgcolor=lightgray,
    fontsize=\footnotesize,
    linenos
    ]
{javascript}
// returns min, max
def intConcat(s:String): (Long, Long) = {
    val l = s.split(" ").permutations.map(_.mkString.toLong).toList
    (l.sorted.head, l.sorted.reverse.head)
}
\end{minted}
\end{document}
